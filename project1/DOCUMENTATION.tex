\documentclass[12pt,a4paper]{article}
\usepackage[utf8]{inputenc}
\usepackage[spanish]{babel}
\usepackage{graphicx}
\usepackage{xcolor}
\usepackage{listings}
\usepackage{geometry}
\usepackage{hyperref}
\usepackage{fancyhdr}
\usepackage{titlesec}
\usepackage{enumitem}
\usepackage{tcolorbox}
\usepackage{fontawesome5}

\geometry{margin=1in}

% Colores personalizados
\definecolor{primarycolor}{RGB}{13,110,253}
\definecolor{successcolor}{RGB}{25,135,84}
\definecolor{codebackground}{RGB}{245,245,245}
\definecolor{codecomment}{RGB}{106,153,85}

% Configuración de hyperlinks
\hypersetup{
    colorlinks=true,
    linkcolor=primarycolor,
    filecolor=primarycolor,
    urlcolor=primarycolor,
    citecolor=primarycolor
}

% Configuración de listings para código
\lstset{
    backgroundcolor=\color{codebackground},
    basicstyle=\ttfamily\small,
    breaklines=true,
    commentstyle=\color{codecomment},
    keywordstyle=\color{primarycolor}\bfseries,
    numbers=left,
    numberstyle=\tiny\color{gray},
    frame=single,
    rulecolor=\color{gray!30},
    tabsize=4
}

% Encabezado y pie de página
\pagestyle{fancy}
\fancyhf{}
\fancyhead[L]{\textbf{Study Groups MVP}}
\fancyhead[R]{\textit{Documentación Técnica}}
\fancyfoot[C]{\thepage}

% Formato de secciones
\titleformat{\section}
{\color{primarycolor}\normalfont\Large\bfseries}
{\thesection}{1em}{}

\titleformat{\subsection}
{\color{primarycolor}\normalfont\large\bfseries}
{\thesubsection}{1em}{}

% Cajas personalizadas
\newtcolorbox{infobox}{
    colback=blue!5!white,
    colframe=primarycolor,
    title=\faInfoCircle\ Información
}

\newtcolorbox{successbox}{
    colback=green!5!white,
    colframe=successcolor,
    title=\faCheckCircle\ Ventajas
}

\begin{document}

% Portada
\begin{titlepage}
    \centering
    \vspace*{2cm}
    
    {\Huge\bfseries Study Groups MVP\par}
    \vspace{1cm}
    {\Large Plataforma Colaborativa de Grupos de Estudio\par}
    \vspace{2cm}
    
    {\large\textit{Documentación Técnica Completa}\par}
    \vspace{1cm}
    
    {\large Versión 1.0\par}
    \vfill
    
    {\large Desarrollado con Django Framework\par}
    \vspace{0.5cm}
    {\large\today\par}
\end{titlepage}

% Índice
\tableofcontents
\newpage

% Resumen ejecutivo
\section*{Resumen Ejecutivo}
\addcontentsline{toc}{section}{Resumen Ejecutivo}

\textbf{Study Groups MVP} es una plataforma web colaborativa diseñada para estudiantes universitarios que desean organizar y participar en grupos de estudio. La plataforma facilita la comunicación, el intercambio de recursos y la planificación de sesiones de estudio, todo en un entorno centralizado y fácil de usar.

\begin{infobox}
Este proyecto sigue la metodología \textbf{MVP (Minimum Viable Product)} al implementar las funcionalidades esenciales que resuelven el problema central: conectar estudiantes para estudiar juntos de manera efectiva.
\end{infobox}

\subsection*{Stack Tecnológico}
\begin{itemize}[leftmargin=*]
    \item \textbf{Backend:} Django 5.2 (Python)
    \item \textbf{Frontend:} Bootstrap 5, JavaScript (Vanilla)
    \item \textbf{Base de Datos:} SQLite (desarrollo), SQL Server (producción)
    \item \textbf{Iconos:} Font Awesome 5
    \item \textbf{Autenticación:} Django Authentication System
\end{itemize}

\newpage

\section{Visión General del Proyecto}

\subsection{Concepto MVP y Justificación}

El concepto de \textbf{Minimum Viable Product (MVP)} se centra en desarrollar un producto con las características mínimas necesarias para satisfacer a los early adopters y validar la idea de negocio.

\subsubsection{Características del MVP}

\begin{enumerate}[leftmargin=*]
    \item \textbf{Funcionalidad Mínima Viable:}
    \begin{itemize}
        \item Crear y unirse a grupos de estudio
        \item Compartir recursos y materiales
        \item Comunicarse mediante comentarios
        \item Programar sesiones de estudio
    \end{itemize}
    
    \item \textbf{Validación Rápida:}
    \begin{itemize}
        \item Permite probar la idea con usuarios reales
        \item Recopila feedback para iteraciones futuras
        \item Minimiza el tiempo de desarrollo inicial
    \end{itemize}
    
    \item \textbf{Base para Iteración:}
    \begin{itemize}
        \item Arquitectura escalable para nuevas features
        \item Código modular y mantenible
        \item Diseño preparado para crecimiento
    \end{itemize}
\end{enumerate}

\subsection{Problema que Resuelve}

Los estudiantes universitarios enfrentan varios desafíos al intentar formar grupos de estudio efectivos:

\begin{itemize}[leftmargin=*]
    \item \textbf{Desorganización:} WhatsApp groups caóticos, emails dispersos
    \item \textbf{Falta de centralización:} Recursos compartidos en múltiples plataformas
    \item \textbf{Dificultad de coordinación:} Programar sesiones sin herramientas adecuadas
    \item \textbf{Comunicación ineficiente:} Sin historial searchable o threading
\end{itemize}

\begin{successbox}
Study Groups MVP centraliza todas estas necesidades en una sola plataforma intuitiva y fácil de usar.
\end{successbox}

\newpage

\section{Módulos Implementados}

\subsection{Módulo 1: Autenticación y Usuarios}

\subsubsection{Sistema de Registro}

\textbf{Funcionalidades implementadas:}
\begin{itemize}[leftmargin=*]
    \item Formulario de registro con validación de campos
    \item Creación automática de perfil al registrarse
    \item Login automático después del registro exitoso
    \item Validación de contraseñas seguras
    \item Prevención de usuarios duplicados
\end{itemize}

\textbf{Implementación técnica:}
\begin{lstlisting}[language=Python,caption=UserRegistrationForm]
class UserRegistrationForm(UserCreationForm):
    email = forms.EmailField(required=True)
    
    class Meta:
        model = User
        fields = ['username', 'email', 'password1', 'password2']
\end{lstlisting}

\begin{successbox}
\textbf{Ventajas:}
\begin{itemize}
    \item Experiencia de usuario fluida (no requiere login post-registro)
    \item Validación robusta previene errores comunes
    \item Seguridad integrada con Django's authentication
\end{itemize}
\end{successbox}

\subsubsection{Sistema de Login/Logout}

\textbf{Características:}
\begin{itemize}[leftmargin=*]
    \item Login con username y contraseña
    \item Logout seguro con redirección configurable
    \item Protección de rutas con decoradores \texttt{@login\_required}
    \item Redirección inteligente a página solicitada (\texttt{?next=})
    \item Session management automático
    \item CSRF protection integrada
\end{itemize}

\subsubsection{Perfiles de Usuario}

Cada usuario tiene un perfil extendido que se crea automáticamente mediante Django signals:

\textbf{Campos del perfil:}
\begin{itemize}[leftmargin=*]
    \item \textbf{Bio:} Descripción personal del estudiante
    \item \textbf{Major:} Carrera o programa de estudios
    \item \textbf{Interests:} Áreas de interés académico
    \item \textbf{Timestamps:} Created\_at, Updated\_at
\end{itemize}

\begin{lstlisting}[language=Python,caption=Modelo Profile]
class Profile(models.Model):
    user = models.OneToOneField(User, on_delete=models.CASCADE)
    bio = models.TextField(max_length=500, blank=True)
    major = models.CharField(max_length=100, blank=True)
    interests = models.TextField(max_length=500, blank=True)
    created_at = models.DateTimeField(auto_now_add=True)
    updated_at = models.DateTimeField(auto_now=True)
\end{lstlisting}

\newpage

\subsection{Módulo 2: Grupos de Estudio}

\subsubsection{Creación de Grupos}

\textbf{Proceso de creación:}
\begin{enumerate}[leftmargin=*]
    \item Usuario autenticado accede a formulario de creación
    \item Completa campos obligatorios: nombre, descripción, materia
    \item Define capacidad máxima de miembros
    \item Sistema asigna automáticamente rol de \textbf{Admin} al creador
    \item Grupo queda visible en búsquedas y listados
\end{enumerate}

\textbf{Campos del modelo StudyGroup:}
\begin{itemize}[leftmargin=*]
    \item \texttt{name} - Nombre del grupo
    \item \texttt{description} - Descripción detallada
    \item \texttt{subject} - Foreign Key a Subject
    \item \texttt{created\_by} - Foreign Key al creador
    \item \texttt{members} - ManyToMany through GroupMembership
    \item \texttt{max\_members} - Límite de capacidad
    \item \texttt{is\_active} - Estado del grupo
    \item \texttt{created\_at, updated\_at} - Timestamps
\end{itemize}

\begin{successbox}
\textbf{Ventajas:}
\begin{itemize}
    \item Organización por materias facilita descubrimiento
    \item Control de capacidad previene grupos sobrepoblados
    \item Roles automáticos simplifican gestión inicial
    \item Flag \texttt{is\_active} permite archivar sin eliminar
\end{itemize}
\end{successbox}

\subsubsection{Sistema de Membresías}

El sistema de membresías implementa una relación many-to-many explícita con campos adicionales:

\textbf{Roles disponibles:}
\begin{itemize}[leftmargin=*]
    \item \textbf{Member:} Rol básico, puede participar en todas las actividades
    \item \textbf{Moderator:} Puede moderar contenido y ayudar en gestión
    \item \textbf{Admin:} Control total del grupo, asigna roles
\end{itemize}

\textbf{Funcionalidades:}
\begin{itemize}[leftmargin=*]
    \item Unirse a grupos con un click (si hay espacio disponible)
    \item Dejar grupos voluntariamente
    \item Protección: último admin no puede abandonar grupo
    \item Validación de capacidad antes de unirse
    \item Tracking de fecha de ingreso (\texttt{joined\_at})
\end{itemize}

\begin{lstlisting}[language=Python,caption=Vista join\_group]
@login_required
def join_group(request, pk):
    group = get_object_or_404(StudyGroup, pk=pk)
    if group.members.count() >= group.max_members:
        messages.error(request, 'This group is already full.')
        return redirect('core:group_detail', pk=pk)
    
    if not group.members.filter(id=request.user.id).exists():
        GroupMembership.objects.create(
            user=request.user, group=group, role='member')
        messages.success(request, 'Successfully joined!')
    return redirect('core:group_detail', pk=pk)
\end{lstlisting}

\subsubsection{Búsqueda y Filtrado}

\textbf{Métodos de búsqueda implementados:}

\begin{enumerate}[leftmargin=*]
    \item \textbf{Búsqueda por texto:}
    \begin{itemize}
        \item Busca en nombre del grupo
        \item Busca en descripción
        \item Busca en nombre de materia asociada
    \end{itemize}
    
    \item \textbf{Filtrado por materia:}
    \begin{itemize}
        \item Dropdown con todas las materias disponibles
        \item Filtro combinable con búsqueda por texto
    \end{itemize}
    
    \item \textbf{Optimizaciones:}
    \begin{itemize}
        \item Paginación (12 grupos por página)
        \item Solo grupos activos (\texttt{is\_active=True})
        \item Orden descendente por fecha de creación
    \end{itemize}
\end{enumerate}

\begin{lstlisting}[language=Python,caption=Vista de búsqueda]
class StudyGroupSearchView(ListView):
    model = StudyGroup
    template_name = 'core/search_results.html'
    context_object_name = 'groups'
    paginate_by = 12

    def get_queryset(self):
        query = self.request.GET.get('q', '')
        subject = self.request.GET.get('subject', '')
        queryset = StudyGroup.objects.filter(is_active=True)
        
        if query:
            queryset = queryset.filter(
                Q(name__icontains=query) |
                Q(description__icontains=query) |
                Q(subject__name__icontains=query))
        if subject:
            queryset = queryset.filter(subject_id=subject)
        return queryset.order_by('-created_at')
\end{lstlisting}

\newpage

\subsection{Módulo 3: Sesiones de Estudio}

\subsubsection{Programación de Sesiones}

Las sesiones de estudio permiten a los grupos coordinar encuentros:

\textbf{Campos del modelo StudySession:}
\begin{itemize}[leftmargin=*]
    \item \texttt{group} - Foreign Key al grupo
    \item \texttt{title} - Título de la sesión
    \item \texttt{description} - Detalles de lo que se estudiará
    \item \texttt{date} - Fecha de la sesión
    \item \texttt{start\_time, end\_time} - Horario
    \item \texttt{location} - Ubicación física
    \item \texttt{is\_online} - Boolean para sesiones virtuales
    \item \texttt{meeting\_link} - URL para sesiones online
    \item \texttt{status} - Estado (Scheduled, In Progress, Completed, Cancelled)
    \item \texttt{created\_by} - Quien creó la sesión
\end{itemize}

\textbf{Estados posibles:}
\begin{itemize}[leftmargin=*]
    \item \textbf{Scheduled:} Sesión programada, pendiente
    \item \textbf{In Progress:} Sesión en curso
    \item \textbf{Completed:} Sesión finalizada
    \item \textbf{Cancelled:} Sesión cancelada
\end{itemize}

\begin{successbox}
\textbf{Ventajas:}
\begin{itemize}
    \item Soporte dual: presencial y online
    \item Tracking de estado facilita seguimiento
    \item Timestamps permiten análisis de asistencia
    \item Integración futura con calendarios (Google, Outlook)
\end{itemize}
\end{successbox}

\subsubsection{Visualización de Sesiones}

En la página de detalle de cada grupo se muestran:

\begin{itemize}[leftmargin=*]
    \item Próximas 5 sesiones programadas
    \item Ordenadas por fecha y hora
    \item Solo sesiones con estado "Scheduled"
    \item Indicador visual de modalidad (online vs presencial)
    \item Link directo a reunión online si aplica
\end{itemize}

\newpage

\subsection{Módulo 4: Materiales de Estudio}

\subsubsection{Compartir Recursos}

Los miembros pueden compartir dos tipos de recursos:

\begin{enumerate}[leftmargin=*]
    \item \textbf{Archivos locales:}
    \begin{itemize}
        \item PDFs, documentos de Word, presentaciones
        \item Imágenes (diagramas, mapas conceptuales)
        \item Código fuente (ZIP, scripts)
        \item Upload directo con Django FileField
    \end{itemize}
    
    \item \textbf{Links externos:}
    \begin{itemize}
        \item Videos de YouTube
        \item Artículos académicos online
        \item Recursos web interactivos
        \item Repositorios de GitHub
    \end{itemize}
\end{enumerate}

\textbf{Modelo StudyMaterial:}
\begin{lstlisting}[language=Python]
class StudyMaterial(models.Model):
    group = models.ForeignKey(StudyGroup, on_delete=models.CASCADE)
    title = models.CharField(max_length=200)
    description = models.TextField()
    file = models.FileField(upload_to='study_materials/', 
                           null=True, blank=True)
    link = models.URLField(null=True, blank=True)
    uploaded_by = models.ForeignKey(User, on_delete=models.CASCADE)
    created_at = models.DateTimeField(auto_now_add=True)
    updated_at = models.DateTimeField(auto_now=True)
\end{lstlisting}

\begin{successbox}
\textbf{Ventajas:}
\begin{itemize}
    \item Centraliza recursos del grupo
    \item Flexibilidad: archivos locales o URLs
    \item Crédito al contribuidor (uploaded\_by)
    \item Timestamps para ordenar por recencia
\end{itemize}
\end{successbox}

\newpage

\subsection{Módulo 5: Sistema de Comentarios}

\subsubsection{Comentarios en Grupos}

Sistema completo de discusión con threading de respuestas:

\textbf{Características principales:}
\begin{itemize}[leftmargin=*]
    \item Publicar comentarios en la página del grupo
    \item Editor de texto con formato preservado (linebreaks)
    \item Solo usuarios autenticados pueden comentar
    \item Timestamp relativo ("5 minutes ago", "2 hours ago")
    \item Indicador visual de comentarios editados
    \item Avatares de usuario (placeholder con posibilidad de personalización)
\end{itemize}

\textbf{Modelo Comment:}
\begin{lstlisting}[language=Python]
class Comment(models.Model):
    group = models.ForeignKey(StudyGroup, 
                             on_delete=models.CASCADE,
                             related_name='comments')
    author = models.ForeignKey(User, on_delete=models.CASCADE)
    content = models.TextField()
    parent = models.ForeignKey('self', on_delete=models.CASCADE,
                              null=True, blank=True,
                              related_name='replies')
    created_at = models.DateTimeField(auto_now_add=True)
    updated_at = models.DateTimeField(auto_now=True)
    is_edited = models.BooleanField(default=False)

    class Meta:
        ordering = ['created_at']

    def save(self, *args, **kwargs):
        if self.pk:  # Si ya existe (edición)
            self.is_edited = True
        super().save(*args, **kwargs)
\end{lstlisting}

\subsubsection{Sistema de Respuestas (Replies)}

Implementación de threading para conversaciones anidadas:

\textbf{Funcionalidades:}
\begin{itemize}[leftmargin=*]
    \item Responder a comentarios específicos
    \item Estructura anidada (comentario → respuestas)
    \item Indentación visual para claridad
    \item Formulario dinámico generado con JavaScript
    \item Self-referencing Foreign Key (\texttt{parent})
\end{itemize}

\textbf{Ventajas del threading:}
\begin{itemize}[leftmargin=*]
    \item Conversaciones contextualizadas
    \item Seguimiento claro de discusiones
    \item Similar a Reddit/Stack Overflow
    \item Previene confusión en discusiones largas
\end{itemize}

\subsubsection{Edición y Eliminación}

Control completo sobre contenido propio:

\textbf{Funcionalidades de edición:}
\begin{enumerate}[leftmargin=*]
    \item Click en "Edit" abre formulario inline
    \item Textarea precargado con contenido actual
    \item Botones: "Save Changes" y "Cancel"
    \item Cancel restaura contenido original sin reload
    \item Flag \texttt{is\_edited} se activa automáticamente
    \item Indicador "(edited)" visible a otros usuarios
\end{enumerate}

\textbf{Funcionalidades de eliminación:}
\begin{itemize}[leftmargin=*]
    \item Confirmación antes de eliminar (JavaScript alert)
    \item Eliminación en cascada de replies (on\_delete=CASCADE)
    \item Solo el autor puede eliminar su comentario
    \item Validación server-side de ownership
\end{itemize}

\textbf{Seguridad implementada:}
\begin{lstlisting}[language=Python,caption=Vista de edición con validación]
@login_required
def edit_comment(request, group_id, comment_id):
    comment = get_object_or_404(Comment, pk=comment_id,
                               author=request.user)
    if request.method == 'POST':
        content = request.POST.get('content')
        if content:
            comment.content = content
            comment.save()  # is_edited se activa automáticamente
            messages.success(request, 'Comment updated!')
    return redirect('core:group_detail', pk=group_id)
\end{lstlisting}

\begin{successbox}
\textbf{Seguridad del sistema de comentarios:}
\begin{itemize}
    \item Validación de ownership server-side
    \item CSRF protection en todos los formularios
    \item XSS prevention con auto-escaping de Django
    \item Login required para todas las acciones
\end{itemize}
\end{successbox}

\subsubsection{Interfaz de Usuario}

Diseño moderno e intuitivo:

\textbf{Componentes UI:}
\begin{itemize}[leftmargin=*]
    \item \textbf{Dropdown menu:} Acciones (Edit/Delete) solo para autor
    \item \textbf{Reply button:} Inline para cada comentario
    \item \textbf{Formularios dinámicos:} Se insertan con JavaScript
    \item \textbf{Avatares:} Placeholders con posibilidad de personalización
    \item \textbf{Timestamps relativos:} Django's \texttt{timesince} filter
    \item \textbf{Badge de contador:} Muestra total de comentarios
\end{itemize}

\textbf{Stack tecnológico UI:}
\begin{itemize}[leftmargin=*]
    \item Bootstrap 5 dropdowns y cards
    \item JavaScript vanilla (no jQuery)
    \item Event delegation para performance
    \item Django template tags para URLs seguras
    \item Font Awesome icons
\end{itemize}

\newpage

\subsection{Módulo 6: Notificaciones}

\subsubsection{Sistema de Notificaciones}

Infraestructura completa para comunicación:

\textbf{Tipos de notificaciones:}
\begin{enumerate}[leftmargin=*]
    \item \textbf{Session Reminder:} Recordatorio de sesión próxima
    \item \textbf{New Material:} Nuevo material compartido en grupo
    \item \textbf{New Comment:} Nuevo comentario en grupo
    \item \textbf{Group Update:} Cambios importantes en el grupo
\end{enumerate}

\textbf{Modelo Notification:}
\begin{lstlisting}[language=Python]
class Notification(models.Model):
    NOTIFICATION_TYPES = [
        ('session_reminder', 'Study Session Reminder'),
        ('new_material', 'New Study Material'),
        ('new_comment', 'New Comment'),
        ('group_update', 'Group Update'),
    ]
    
    recipient = models.ForeignKey(User, 
                                 on_delete=models.CASCADE,
                                 related_name='notifications')
    notification_type = models.CharField(
        max_length=20, choices=NOTIFICATION_TYPES)
    group = models.ForeignKey(StudyGroup, 
                             on_delete=models.CASCADE)
    title = models.CharField(max_length=200)
    message = models.TextField()
    related_link = models.URLField(blank=True)
    is_read = models.BooleanField(default=False)
    created_at = models.DateTimeField(auto_now_add=True)

    class Meta:
        ordering = ['-created_at']
\end{lstlisting}

\subsubsection{Email Notifications}

Sistema de recordatorios por email:

\textbf{Funcionalidades:}
\begin{itemize}[leftmargin=*]
    \item Envío automático de recordatorios de sesiones
    \item Templates HTML profesionales
    \item Información completa: fecha, hora, ubicación/link
    \item Envío asíncrono (no bloquea request)
    \item Fallback a plain text si HTML no soportado
\end{itemize}

\begin{lstlisting}[language=Python,caption=Envío de email notification]
@classmethod
def send_session_reminder(cls, session):
    for member in session.group.members.all():
        # Crear notificación in-app
        notification = cls.objects.create(
            recipient=member,
            notification_type='session_reminder',
            group=session.group,
            title=f'Upcoming Session: {session.title}',
            message=f'Session on {session.date} at {session.start_time}'
        )
        
        # Enviar email
        html_message = render_to_string(
            'core/email/session_reminder.html',
            {'user': member, 'session': session})
        send_mail(
            subject=f'Reminder: {session.title}',
            message=strip_tags(html_message),
            html_message=html_message,
            from_email=settings.DEFAULT_FROM_EMAIL,
            recipient_list=[member.email],
            fail_silently=True
        )
\end{lstlisting}

\begin{successbox}
\textbf{Ventajas del sistema de notificaciones:}
\begin{itemize}
    \item Mejora engagement y retención
    \item Reduce no-shows a sesiones
    \item Mantiene usuarios informados
    \item Extensible a notificaciones push (futuro)
\end{itemize}
\end{successbox}

\newpage

\subsection{Módulo 7: Interfaz y Experiencia de Usuario}

\subsubsection{Design System}

Sistema de diseño consistente en toda la aplicación:

\textbf{Framework y librerías:}
\begin{itemize}[leftmargin=*]
    \item \textbf{Bootstrap 5:} Framework CSS responsive
    \item \textbf{Font Awesome 5:} 1500+ iconos vectoriales
    \item \textbf{Custom CSS:} Sobreescrituras para branding
    \item \textbf{Google Fonts:} Tipografía profesional
\end{itemize}

\textbf{Principios de diseño aplicados:}
\begin{enumerate}[leftmargin=*]
    \item \textbf{Mobile-first:} Diseño responsive desde pantallas pequeñas
    \item \textbf{Accesibilidad:} ARIA labels, contraste adecuado, navegación por teclado
    \item \textbf{Consistencia:} Mismos colores, espaciados y componentes
    \item \textbf{Feedback visual:} Messages, loading states, hover effects
\end{enumerate}

\subsubsection{Templates y Layouts}

Arquitectura de templates DRY (Don't Repeat Yourself):

\textbf{Estructura de templates:}
\begin{itemize}[leftmargin=*]
    \item \textbf{base.html:} Layout principal con navbar, footer, blocks
    \item \textbf{Templates por módulo:} group\_list, group\_detail, profile, etc.
    \item \textbf{Partials reutilizables:} comment\_section, materials\_section
    \item \textbf{Admin templates:} Customización del Django admin
\end{itemize}

\textbf{Django Template Language features utilizados:}
\begin{itemize}[leftmargin=*]
    \item Template inheritance (\texttt{\{% extends \%\}})
    \item Template blocks (\texttt{\{% block content \%\}})
    \item Include tags (\texttt{\{% include 'partial.html' \%\}})
    \item Template filters (\texttt{|timesince}, \texttt{|linebreaks})
    \item Custom template tags (si aplica)
\end{itemize}

\subsubsection{Navegación}

Sistema de navegación intuitivo:

\textbf{Componentes de navegación:}
\begin{enumerate}[leftmargin=*]
    \item \textbf{Navbar principal:}
    \begin{itemize}
        \item Logo/Home link
        \item Browse Groups
        \item My Groups (si autenticado)
        \item Search bar
        \item User dropdown (Profile, Logout)
    \end{itemize}
    
    \item \textbf{Breadcrumbs:}
    \begin{itemize}
        \item Home > Groups > Django Study Group
        \item Orientación contextual clara
    \end{itemize}
    
    \item \textbf{Footer:}
    \begin{itemize}
        \item Links útiles (About, Contact, Terms)
        \item Información de copyright
    \end{itemize}
\end{enumerate}

\begin{successbox}
\textbf{Ventajas de la UI/UX:}
\begin{itemize}
    \item Curva de aprendizaje mínima
    \item Familiar para usuarios de redes sociales
    \item Performance rápida (< 2 segundos load time)
    \item Compatible con lectores de pantalla
\end{itemize}
\end{successbox}

\newpage

\subsection{Módulo 8: Seguridad y Validaciones}

\subsubsection{Autenticación y Autorización}

Múltiples capas de seguridad:

\textbf{Decoradores y mixins:}
\begin{lstlisting}[language=Python]
# Function-based views
@login_required
def create_group(request):
    # Solo usuarios autenticados
    pass

# Class-based views
class StudyGroupCreateView(LoginRequiredMixin, CreateView):
    # Solo usuarios autenticados
    pass
\end{lstlisting}

\textbf{Validación de ownership:}
\begin{lstlisting}[language=Python]
@login_required
def edit_comment(request, group_id, comment_id):
    # get_object_or_404 valida que comment exista
    # author=request.user valida que sea el dueño
    comment = get_object_or_404(Comment, 
                               pk=comment_id,
                               author=request.user)
    # Solo el autor puede llegar aquí
\end{lstlisting}

\textbf{Protecciones implementadas:}
\begin{itemize}[leftmargin=*]
    \item CSRF tokens en todos los formularios POST
    \item SQL Injection prevention (Django ORM)
    \item XSS prevention (auto-escaping de templates)
    \item Clickjacking protection (X-Frame-Options)
    \item Session security (HttpOnly cookies)
\end{itemize}

\subsubsection{Validaciones de Datos}

Validaciones a nivel de modelo, form y view:

\textbf{Nivel de modelo:}
\begin{lstlisting}[language=Python]
class GroupMembership(models.Model):
    user = models.ForeignKey(User, on_delete=models.CASCADE)
    group = models.ForeignKey(StudyGroup, on_delete=models.CASCADE)
    role = models.CharField(max_length=20, choices=ROLE_CHOICES)
    
    class Meta:
        unique_together = ['user', 'group']  # No duplicados
\end{lstlisting}

\textbf{Nivel de form:}
\begin{lstlisting}[language=Python]
class UserRegistrationForm(UserCreationForm):
    email = forms.EmailField(required=True)
    
    def clean_email(self):
        email = self.cleaned_data.get('email')
        if User.objects.filter(email=email).exists():
            raise ValidationError("Email already registered")
        return email
\end{lstlisting}

\textbf{Nivel de view:}
\begin{lstlisting}[language=Python]
@login_required
def join_group(request, pk):
    group = get_object_or_404(StudyGroup, pk=pk)
    
    # Validar capacidad
    if group.members.count() >= group.max_members:
        messages.error(request, 'Group is full')
        return redirect('core:group_detail', pk=pk)
    
    # Validar no duplicado
    if group.members.filter(id=request.user.id).exists():
        messages.warning(request, 'Already a member')
        return redirect('core:group_detail', pk=pk)
    
    # Todo OK, crear membership
    GroupMembership.objects.create(...)
\end{lstlisting}

\subsubsection{Manejo de Errores}

Framework de mensajes para feedback consistente:

\textbf{Tipos de mensajes:}
\begin{itemize}[leftmargin=*]
    \item \textbf{Success:} Acción completada exitosamente
    \item \textbf{Info:} Información neutral
    \item \textbf{Warning:} Advertencia, acción no crítica
    \item \textbf{Error:} Error que previno acción
\end{itemize}

\begin{lstlisting}[language=Python]
from django.contrib import messages

# Success message
messages.success(request, 'Comment posted successfully!')

# Error message
messages.error(request, 'This group is already full.')

# Warning message
messages.warning(request, 'You are already a member.')

# Info message
messages.info(request, 'Please log in to continue.')
\end{lstlisting}

\newpage

\subsection{Módulo 9: Base de Datos y Exportación}

\subsubsection{Diseño de Base de Datos}

Arquitectura relacional normalizada:

\textbf{Diagrama de relaciones principales:}

\begin{itemize}[leftmargin=*]
    \item \textbf{User} ← (1:1) → \textbf{Profile}
    \item \textbf{User} ← (1:N) → \textbf{StudyGroup} (created\_by)
    \item \textbf{User} ← (M:N) → \textbf{StudyGroup} (through GroupMembership)
    \item \textbf{StudyGroup} ← (1:N) → \textbf{StudySession}
    \item \textbf{StudyGroup} ← (1:N) → \textbf{StudyMaterial}
    \item \textbf{StudyGroup} ← (1:N) → \textbf{Comment}
    \item \textbf{Comment} ← (1:N) → \textbf{Comment} (parent/replies)
    \item \textbf{Subject} ← (1:N) → \textbf{StudyGroup}
\end{itemize}

\textbf{Políticas de eliminación (on\_delete):}
\begin{itemize}[leftmargin=*]
    \item \texttt{CASCADE}: Profile, Memberships, Sessions, Materials, Comments
    \item \texttt{SET\_NULL}: Para referencias opcionales (created\_by en Session)
    \item \texttt{PROTECT}: Para prevenir eliminación accidental (Subject)
\end{itemize}

\textbf{Indexes y optimizaciones:}
\begin{lstlisting}[language=Python]
class Comment(models.Model):
    # ...campos...
    
    class Meta:
        ordering = ['created_at']  # Default order
        indexes = [
            models.Index(fields=['group', 'created_at']),
            models.Index(fields=['author']),
        ]
\end{lstlisting}

\subsubsection{Exportación a SQL Server}

Sistema completo de exportación para analytics:

\textbf{Proceso de exportación:}
\begin{enumerate}[leftmargin=*]
    \item \textbf{Generación de datos:}
    \begin{lstlisting}[language=Bash]
python create_samples.py
    \end{lstlisting}
    Crea usuarios, grupos, sesiones, materiales y comentarios de prueba.
    
    \item \textbf{Export a CSV:}
    \begin{itemize}
        \item Django management command personalizado
        \item Exports automáticos a carpeta \texttt{exports/}
        \item CSVs para cada modelo
    \end{itemize}
    
    \item \textbf{Conversión a TSV:}
    \begin{lstlisting}[language=PowerShell]
cd exports
.\convert_csvs_to_tsv.ps1
    \end{lstlisting}
    Convierte CSVs a TSV para importar a SQL Server sin problemas de comillas.
    
    \item \textbf{Importación a SQL Server:}
    \begin{lstlisting}[language=SQL]
-- En SSMS
USE StudyGroupsMVP;
GO
-- Ejecutar import_to_sqlserver.sql
    \end{lstlisting}
\end{enumerate}

\textbf{Script de importación SQL (características):}
\begin{itemize}[leftmargin=*]
    \item Crea schema \texttt{stg} para staging tables
    \item BULK INSERT desde TSV files
    \item Mapeo de IDs legacy a nuevos IDs IDENTITY
    \item Preservación de relaciones parent-child (Comments)
    \item Constraints y Foreign Keys
    \item Transacciones para integridad
\end{itemize}

\begin{successbox}
\textbf{Ventajas de exportación SQL Server:}
\begin{itemize}
    \item Análisis con Power BI / Tableau
    \item Reportes avanzados con T-SQL
    \item Backup enterprise-grade
    \item Interoperabilidad con ERP/CRM
\end{itemize}
\end{successbox}

\newpage

\subsection{Módulo 10: Herramientas y Utilidades}

\subsubsection{Admin Panel Customizado}

Django Admin configurado para gestión eficiente:

\textbf{Customizaciones implementadas:}
\begin{lstlisting}[language=Python,caption=Admin para StudyGroup]
@admin.register(StudyGroup)
class StudyGroupAdmin(admin.ModelAdmin):
    list_display = ['name', 'subject', 'created_by', 
                   'max_members', 'is_active', 'created_at']
    list_filter = ['subject', 'is_active', 'created_at']
    search_fields = ['name', 'description', 
                    'created_by__username']
    date_hierarchy = 'created_at'
    
    # Inline editing de memberships
    inlines = [GroupMembershipInline]
\end{lstlisting}

\textbf{Features del admin:}
\begin{itemize}[leftmargin=*]
    \item Filtros por fecha, materia, estado
    \item Búsqueda en múltiples campos
    \item Inline editing para relaciones
    \item Bulk actions (activar/desactivar grupos en masa)
    \item Custom actions (enviar notificaciones, exportar)
\end{itemize}

\subsubsection{Fixtures y Datos Iniciales}

Sistema de carga de datos predefinidos:

\textbf{Fixture: initial\_subjects.json}
\begin{lstlisting}[language=JSON]
[
  {
    "model": "core.subject",
    "pk": 1,
    "fields": {
      "name": "Mathematics",
      "description": "Algebra, Calculus, Statistics"
    }
  },
  {
    "model": "core.subject",
    "pk": 2,
    "fields": {
      "name": "Computer Science",
      "description": "Programming, Algorithms, Data Structures"
    }
  }
  // ... más materias
]
\end{lstlisting}

\textbf{Comando de carga:}
\begin{lstlisting}[language=Bash]
python manage.py loaddata core/fixtures/initial_subjects.json
\end{lstlisting}

\textbf{Script create\_samples.py:}
\begin{itemize}[leftmargin=*]
    \item Crea 10+ usuarios de prueba
    \item Genera 15+ grupos en diferentes materias
    \item Programa 20+ sesiones de estudio
    \item Comparte 30+ materiales
    \item Publica 50+ comentarios con replies
    \item Útil para demos y testing
\end{itemize}

\newpage

\section{Ventajas Competitivas del MVP}

\subsection{Ventajas Técnicas}

\begin{enumerate}[leftmargin=*]
    \item \textbf{Arquitectura Escalable}
    \begin{itemize}
        \item Django ORM abstrae la base de datos
        \item Migración fácil a PostgreSQL/MySQL
        \item Template system permite rebranding sin cambios de backend
        \item API REST agregable con Django REST Framework
    \end{itemize}
    
    \item \textbf{Performance Optimizado}
    \begin{itemize}
        \item \texttt{select\_related()} y \texttt{prefetch\_related()} reducen queries N+1
        \item Paginación en todas las vistas largas
        \item Caching preparado con Django Cache Framework
        \item Static files con versioning para cache busting
    \end{itemize}
    
    \item \textbf{Código Mantenible}
    \begin{itemize}
        \item Arquitectura DRY (Don't Repeat Yourself)
        \item Separación de concerns (Models, Views, Templates)
        \item Código auto-documentado con docstrings
        \item Tests unitarios preparados (estructura de carpetas)
    \end{itemize}
    
    \item \textbf{Seguridad Robusta}
    \begin{itemize}
        \item Framework Django con 15+ años de hardening
        \item Protección contra OWASP Top 10
        \item Updates de seguridad regulares de Django
        \item Best practices seguidas en todo el código
    \end{itemize}
\end{enumerate}

\subsection{Ventajas de Negocio}

\begin{enumerate}[leftmargin=*]
    \item \textbf{Time to Market}
    \begin{itemize}
        \item MVP funcional desarrollado en semanas
        \item Deployment rápido (Heroku, Railway, PythonAnywhere)
        \item No requiere infraestructura compleja inicial
        \item Iteraciones rápidas basadas en feedback
    \end{itemize}
    
    \item \textbf{Bajo Costo Operacional}
    \begin{itemize}
        \item Stack 100\% open source (Django, SQLite, Bootstrap)
        \item Hosting gratuito disponible (tiers free de Heroku, Railway)
        \item Sin licencias de software
        \item Escalamiento horizontal disponible
    \end{itemize}
    
    \item \textbf{Validación de Concepto}
    \begin{itemize}
        \item Core features implementadas y validables
        \item Métricas rastreables (grupos creados, miembros activos)
        \item Feedback loops integrados (comments, sessions)
        \item Pivotes rápidos si es necesario
    \end{itemize}
    
    \item \textbf{Monetización Futura}
    \begin{itemize}
        \item Freemium model (grupos ilimitados vs limitados)
        \item Premium features (videollamadas, storage adicional)
        \item Institucional (licencias para universidades)
        \item Ads (no intrusivos, relevantes al contexto académico)
    \end{itemize}
\end{enumerate}

\subsection{Ventajas para Usuarios}

\begin{enumerate}[leftmargin=*]
    \item \textbf{Facilidad de Uso}
    \begin{itemize}
        \item Interfaz familiar similar a redes sociales
        \item Curva de aprendizaje mínima
        \item Mobile-friendly (responsive design)
        \item Accesible desde cualquier navegador
    \end{itemize}
    
    \item \textbf{Centralización}
    \begin{itemize}
        \item Todo en un solo lugar (grupos, materiales, calendario, chat)
        \item No más emails dispersos o WhatsApp groups caóticos
        \item Historial searchable de discusiones
        \item Recursos siempre accesibles
    \end{itemize}
    
    \item \textbf{Colaboración Efectiva}
    \begin{itemize}
        \item Comunicación asíncrona (comments con threading)
        \item Compartir recursos fácilmente
        \item Organización transparente (roles, capacidad)
        \item Notificaciones que mantienen engagement
    \end{itemize}
    
    \item \textbf{Productividad Académica}
    \begin{itemize}
        \item Mejor coordinación de sesiones de estudio
        \item Recursos centralizados mejoran preparación
        \item Discusiones estructuradas facilitan aprendizaje
        \item Tracking de progreso (sesiones completadas)
    \end{itemize}
\end{enumerate}

\newpage

\section{Roadmap: Evolución Post-MVP}

\subsection{Features Corto Plazo (1-3 meses)}

\begin{enumerate}[leftmargin=*]
    \item \textbf{Notificaciones en Tiempo Real}
    \begin{itemize}
        \item WebSockets con Django Channels
        \item Notificaciones push en navegador
        \item Badge count en navbar
    \end{itemize}
    
    \item \textbf{Upload Múltiple de Archivos}
    \begin{itemize}
        \item Drag \& drop interface
        \item Progress bars para uploads
        \item Thumbnails para imágenes
    \end{itemize}
    
    \item \textbf{Sistema de Tags}
    \begin{itemize}
        \item Tags para categorizar grupos (#exam-prep, #final-project)
        \item Filtrado por tags
        \item Auto-suggest tags populares
    \end{itemize}
    
    \item \textbf{Búsqueda Avanzada}
    \begin{itemize}
        \item Múltiples filtros combinables
        \item Búsqueda full-text con Elasticsearch
        \item Guardar búsquedas favoritas
    \end{itemize}
\end{enumerate}

\subsection{Features Mediano Plazo (3-6 meses)}

\begin{enumerate}[leftmargin=*]
    \item \textbf{Integración con Calendarios}
    \begin{itemize}
        \item Sincronización con Google Calendar
        \item Exportar sesiones a .ics
        \item Recordatorios automáticos
    \end{itemize}
    
    \item \textbf{Videollamadas Integradas}
    \begin{itemize}
        \item Integración con Zoom API o Jitsi
        \item Sala de reunión permanente por grupo
        \item Grabación de sesiones (premium)
    \end{itemize}
    
    \item \textbf{Gamificación}
    \begin{itemize}
        \item Sistema de puntos por participación
        \item Badges por logros (10 comentarios, organizar 5 sesiones)
        \item Leaderboards por grupo/universidad
    \end{itemize}
    
    \item \textbf{Recomendaciones con ML}
    \begin{itemize}
        \item Algoritmo de matching automático
        \item Sugerencia de grupos basada en intereses
        \item Predicción de horarios óptimos para sesiones
    \end{itemize}
\end{enumerate}

\subsection{Features Largo Plazo (6-12 meses)}

\begin{enumerate}[leftmargin=*]
    \item \textbf{Mobile Apps Nativas}
    \begin{itemize}
        \item Apps iOS/Android con React Native o Flutter
        \item Push notifications nativas
        \item Offline mode para materiales
    \end{itemize}
    
    \item \textbf{AI/ML Avanzado}
    \begin{itemize}
        \item Chatbot para responder preguntas frecuentes
        \item Resúmenes automáticos de discusiones largas
        \item Detección de contenido inapropiado
    \end{itemize}
    
    \item \textbf{Sistema de Reputación}
    \begin{itemize}
        \item Ratings de miembros (helpfulness, reliability)
        \item Trust scores para combatir spam
        \item Verificación de estudiantes (email institucional)
    \end{itemize}
    
    \item \textbf{Monetización Premium}
    \begin{itemize}
        \item Tier gratuito: 3 grupos, 100MB storage
        \item Tier premium: Grupos ilimitados, 10GB storage, videollamadas
        \item Tier institucional: Panel de analytics, integración LMS
    \end{itemize}
\end{enumerate}

\newpage

\section{Métricas de Éxito del MVP}

\subsection{KPIs Principales}

\begin{enumerate}[leftmargin=*]
    \item \textbf{Adoption Rate}
    \begin{itemize}
        \item Definición: Porcentaje de estudiantes que crean cuenta después de visitar
        \item Objetivo inicial: 15-20\%
        \item Medición: Google Analytics + Django analytics
    \end{itemize}
    
    \item \textbf{Engagement}
    \begin{itemize}
        \item Comentarios promedio por grupo activo
        \item Sesiones programadas por grupo/mes
        \item Materiales compartidos por grupo/mes
        \item Objetivo: 5+ comentarios, 2+ sesiones, 3+ materiales por grupo/mes
    \end{itemize}
    
    \item \textbf{Retention}
    \begin{itemize}
        \item Usuarios que regresan semanalmente
        \item Objetivo: 40\% semana 1, 25\% mes 1
        \item Cohort analysis para tracking
    \end{itemize}
    
    \item \textbf{Growth}
    \begin{itemize}
        \item Nuevos grupos creados mensualmente
        \item Objetivo: 20\% crecimiento mensual
        \item Viral coefficient (invitaciones)
    \end{itemize}
\end{enumerate}

\subsection{Objetivos Iniciales (Primeros 3 meses)}

\begin{itemize}[leftmargin=*]
    \item \checkmark 50+ usuarios registrados
    \item \checkmark 10+ grupos activos
    \item \checkmark 80\% de grupos con al menos 1 sesión programada
    \item \checkmark Promedio de 5+ comentarios por grupo
    \item \checkmark 3+ materiales compartidos por grupo
    \item \checkmark 60\% de usuarios retornan dentro de 7 días
    \item \checkmark NPS (Net Promoter Score) > 40
\end{itemize}

\subsection{Herramientas de Medición}

\begin{enumerate}[leftmargin=*]
    \item \textbf{Google Analytics}
    \begin{itemize}
        \item Tráfico, bounce rate, páginas más visitadas
        \item Conversión de visitante a registro
        \item Demografía y dispositivos
    \end{itemize}
    
    \item \textbf{Mixpanel / Amplitude}
    \begin{itemize}
        \item Event tracking (grupo creado, comentario publicado)
        \item Funnels (registro → crear grupo → primera sesión)
        \item Retention cohorts
    \end{itemize}
    
    \item \textbf{Django Analytics Custom}
    \begin{itemize}
        \item Queries SQL para métricas específicas
        \item Dashboard admin con estadísticas
        \item Exportes periódicos para análisis
    \end{itemize}
    
    \item \textbf{Hotjar / FullStory}
    \begin{itemize}
        \item Session recordings
        \item Heatmaps de clicks
        \item Feedback surveys
    \end{itemize}
\end{enumerate}

\newpage

\section{Conclusión: Por Qué Este MVP Es Exitoso}

\subsection{Cumple con los Principios MVP}

\begin{enumerate}[leftmargin=*]
    \item \textbf{Mínimo}
    \begin{itemize}
        \item Solo features esenciales implementadas
        \item Sin bloat o complejidad innecesaria
        \item Enfoque claro en problema central
    \end{itemize}
    
    \item \textbf{Viable}
    \begin{itemize}
        \item Funciona end-to-end sin bugs críticos
        \item Deployable a producción hoy
        \item Performance aceptable para usuarios iniciales
    \end{itemize}
    
    \item \textbf{Producto}
    \begin{itemize}
        \item No es un prototipo, es software usable
        \item Valor real para usuarios desde día 1
        \item Iteraciones basadas en feedback real
    \end{itemize}
\end{enumerate}

\subsection{Resuelve un Problema Real}

\begin{tcolorbox}[colback=blue!5!white,colframe=blue!75!black,title=Problema → Solución]
\begin{itemize}
    \item \textbf{Coordinación caótica} → Sistema de grupos con membresías y roles
    \item \textbf{Falta de comunicación} → Comentarios con threading y notificaciones
    \item \textbf{Recursos dispersos} → Materiales centralizados con upload/links
    \item \textbf{Planificación difícil} → Sesiones programadas con recordatorios
\end{itemize}
\end{tcolorbox}

\subsection{Base Sólida para Crecimiento}

\begin{enumerate}[leftmargin=*]
    \item \textbf{Arquitectura permite escalar}
    \begin{itemize}
        \item Django ORM facilita cambio de DB
        \item Código modular facilita agregar features
        \item API REST preparada con DRF
    \end{itemize}
    
    \item \textbf{Código mantenible}
    \begin{itemize}
        \item Convenciones Django estándar
        \item Documentación inline
        \item Tests preparados
    \end{itemize}
    
    \item \textbf{Stack moderno}
    \begin{itemize}
        \item Django 5.2 (LTS support hasta 2026)
        \item Bootstrap 5 (actualizable)
        \item JavaScript vanilla (sin deuda técnica de jQuery)
    \end{itemize}
\end{enumerate}

\subsection{Diferenciadores Clave}

\begin{table}[h]
\centering
\begin{tabular}{|l|c|c|}
\hline
\textbf{Feature} & \textbf{Study Groups MVP} & \textbf{Competidores} \\
\hline
Grupos ilimitados & \checkmark & Limitado \\
Comments con threading & \checkmark & Básico \\
Sesiones presenciales + online & \checkmark & Solo online \\
Open source & \checkmark & Propietario \\
Self-hosteable & \checkmark & SaaS only \\
No ads & \checkmark & Con ads \\
\hline
\end{tabular}
\caption{Comparación con plataformas existentes}
\end{table}

\vfill

\begin{center}
\Large\textbf{Study Groups MVP: Conectando estudiantes para el éxito académico colaborativo}
\end{center}

\end{document}
